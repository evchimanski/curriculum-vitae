\documentclass[12pt,a4paper,sans]{moderncv}

%\usepackage[brazil]{babel}    		% Usado para os dicionarios Incompativel com xy package
\usepackage[none]{hyphenat}

\moderncvstyle{casual}
%\moderncvstyle{oldstyle}
%\moderncvstyle{classic}
%\moderncvstyle{banking}
%\moderncvcolor{green}
%\moderncvcolor{purple}
\moderncvcolor{red}
\renewcommand{\familydefault}{\sfdefault}

\newcommand{\latex}{\LaTeX~}

\usepackage[utf8]{inputenc}

\usepackage[scale=0.75]{geometry}

\name{Emanuel Vicente Chimanski}{\small{Chimanski, E. V.; Emanuel V. Chimanski; E. V. Chimanski}}
\title{Postdoctoral Research Staff Member}
%\address{Av: Dr. Nelson DÁvila, 1178, Ap: 21}{12245-030, SJC}{SP}
\address{}{Livemore/CA}{US}

\phone[mobile]{+1~ 925 404 7177}
\email{evchimanski@gmail.com, chimanski1@llnl.gov}
%\quote{Se algo \'e muito dif\'icil de fazer, \\ ent\~ao n\~ao vale a pena ser feito \\ \textbf{Homer Simpson}}

\begin{document}

\makecvtitle

\section{Education and Training}

\cventry{2015--2019}{Ph.D. in Science (Physics)}{Aeronautics Institute of Technology -- ITA}{Thesis: Extension of the Quantum formalism of MSD Reactions}{Advisor: Prof. Dr. Brett V. Carlson (ITA) and Co-Advisor: Dr. Roberto Capote Noy (IAEA)}{%.\newline{}
	%\begin{itemize}
	%	\item Gest\~ao de Projetos de Pesquisa.
	%\end{itemize}
}

\bigskip
\cventry{2013--2015}{Master in Science (Physics)}{Aeronautics Institute of Technology -- ITA}{Thesis: Route to hyperchaos in Rayleigh-B{\'e}nard convection}{Advisor: Prof. Dr. Erico L. Rempel, Co-advisor:
 Dr. Roman Chertovskih}{}
\bigskip
\cventry{2009--2013}{Physics degree}{Universidade Estadual do Centro Oeste -- UNICENTRO}{Thesis: Estat{\'i}stica de n{\'i}veis em bilhares qu{\^a}nticos (Energy level Statistics in Quantum Billiards)}{Advisor: Prof. Dr.
  Eduardo Vicentini}{}

\bigskip
\cventry{2021}{Machine Learning}{Stanford|online}{}{}{}
\bigskip
\cventry{2021}{XX Jorge André Swieca Summer School on Theoretical Nuclear Physics}{online-Brazil}{}{}{}
\bigskip
\cventry{2019}{XIX Jorge André Swieca Summer School on Theoretical Nuclear Physics}{SP-Brazil}{}{}{}
\bigskip
\cventry{2016}{School on Effective Field Theory across Length Scales }{South American Institute for Fundamental Research, ICTP-SAIFR, Brazil}{}{}{} 
\bigskip
\cventry{2016}{School on Physics Applications in Biology at South American Institute for Fundamental Research, ICTP-SAIFR, Brazil}{}{}{}
\bigskip
\cventry{2014}{Topycs in Computational Cosmology at Instituto Nacional de Pesquisas Espaciais}{INPE,Brazil}{}{}{}
\bigskip
\cventry{2009--2015}{English Course}{Wizard Brasil}{}{}{}
\bigskip

\section{Appointments}
\cventry{2019--present}{Postdoctoral Research Staff Member (Physical and Life Sciences Directorate, Nuclear and Chemincal Sciences Division)}{Lawrence Livermore National Laboratory -- LLNL}{Nuclear Structure and Nuclear Reactions}{Supervisor: Dr. Jutta Escher and Dr. Walid Younes. Reporting to Bret Beck.}{%.\newline{}
	%\begin{itemize}
	%	\item Gest\~ao de Projetos de Pesquisa.
	%\end{itemize}
}
\bigskip
\cventry{2018 (Sep-Dec)}{Ph.D. visiting student at Florida State University (FSU -- Tallahassee/Florida-- US ) in the Physics Department}{}{Supervisor: Dr. Alexander Volya}{%.\newline{}
	%\begin{itemize}
	%	\item Gest\~ao de Projetos de Pesquisa.
	%\end{itemize}
}
\bigskip
\cventry{2017--2018}{Ph.D. internship at International Atomic Energy Agency (IAEA -- Vienna/Austria) in the Nuclear Data Development Unit}{}{Supervisor: Dr. Roberto Capote Noy}{%.\newline{}
	%\begin{itemize}
	%	\item Gest\~ao de Projetos de Pesquisa.
	%\end{itemize}
}
\bigskip


\section{Awards and Leadership Roles}
\begin{itemize}
\item My publication “Quasiparticle nature of excited states in random-phase approximation” was selected as Top 10 contributions in the quadrennium of the Post-Graduation Program. Phys. Rev. C 99 014305 (2019). 
\item Vice-chair of the APS chapter at the Lawrence Livermore National Laboratory (LLNL) – 2021. 
\end{itemize}
\bigskip

\section{Languages}

\cvitemwithcomment{Portuguese}{native}{}
\cvitemwithcomment{English}{writing: good, reading: good, speaking: good}{}

\section{Computer skills}
\cventry{}{Operational system}{}{}{}
{
	\begin{itemize}
	\item GNU/Linux.
        \end{itemize}
}
\cventry{}{Programming}{}{}{}
{
	\begin{itemize}
	\item FORTRAN90
        \item  GNU Octave
        \item Python including: numpy, numba, tensorflow, scikit-learn
          \item \LaTeX.
        \end{itemize}
}

\section{Research and work experience}
\cventry{2019 -- present}{Postdoctoral at Lawrance Livermore National Laboratory (LLNL), Livermore/CA -- US}{Surrogate nuclear reactions and deformed nuclei}{}{}{}

\cventry{2018}{Visiting Student at the Florida State University (FSU) (Physics Dep.), Tallahassee/Florida -- US}{Collective states and Random Matrix}{}{}{}
\cventry{2017 -- 2018}{Intern at the International Atomic Energy Agency -- IAEA (Nuclear Data Development Unit), Vienna/Austria}{Nuclear Reaction Models, Pre-Equilibrium Reactions}{}{}{}
\cventry{2013 -- 2019}{Aeronautics Institute of Technology -- ITA, SP/Brazil}{Classical/Quantum Chaos, Mathematical Modelling, Nuclear Physics,  Nuclear Reactions}{}{}{}
\bigskip
\cventry{2011--2013}{Universidade Estadual do Centro Oeste -- UNICENTRO, PR/Brazil}{Quantum chaos and quantum billiards}{}{}{}

\section{Teaching experience}
\cventry{2015}{FIS-14 Physics (mechanics) laboratory}{Assistant teacher under supervision of Prof. Dr. José Silvério Edmundo Germano, Aeronautics Institute of Technology -- ITA}{}{}{}
\cventry{2012}{Fundamental Physics I}{Assistant teacher under supervision of Prof. Dr. Ricardo Yoshimitsu Miyahara, Universidade Estadual do Centro Oeste -- UNICENTRO}{}{}{}

%\bibliographystyle{plain}
%\bibliography{/home/ronaldo/Dropbox/references/library}

\section{Publications}
\cventry{}{Published}{}{}{}
{
  \begin{itemize}
    \item E. V. Chimanski, and  B. V. Carlson. Nucleon-induced inelastic scattering with statistical strength functions and the ECIS direct reaction code. EPJA, (2021).
  \item Chimanski, E.V., Souza, L.A., Carlson, B.V. The S\~ao Paulo Potential and the 3He Breakup Reaction at 130 MeV on 93Nb and 197Au. Braz J Phys 51, 323-327 (2021).
  \item E. V. Chimanski, B. V. Carlson, R. Capote, A J Koning. Quasiparticle nature of excited states in random-phase approximation. \textit{Phys. Rev. C} \textbf{99} 014305 (2019).
  \item E. V. Chimanski, R. Capote, B. V. Carlson and A J Koning. Statistical multi-step direct reaction models and the eikonal approximation \textit{CERN Proceedings series of the 15th edition of the Varenna Conference on Nuclear Reaction Mechanisms}(2018).
  \item E. V. Chimanski, B. V. Carlson, R. Capote and A J Koning. The role of nucleon knockout in pre-equilibrium reactions \textit{CERN Proceedings series of the 15th edition of the Varenna Conference on Nuclear Reaction Mechanisms}(2018).
  \item Hussein, Mahi S. ; Souza, Lucas A. ; Chimanski, Emanuel ; Carlson, Brett; Frederico, Tobias. Inclusive Breakup Theory of Three-Body Halos. \textit {EPJ Web of Conferences} (2017).
  \item R. Chertovskih, E. L. Rempel and E. V. Chimanski. Magnetic field generation by intermittent convection, \textit{PLA} (2017).
  \item R. Chertovskih, E. V. Chimanski and E. L. Rempel. Route to hyperchaos in Rayleigh-B{\'e}nard convection, \textit{EPL}, \textbf{112} (2015) 14001.
  \item Emanuel V. Chimanski, Erico L. Rempel, Roman Chertovskih. On-off intermittency and spatiotemporal chaos in three-dimensional Rayleigh-B{\'e}nard convection,\textit{Advances in Space Research}, \textbf{57} (2016), 1440-1447.
  \end{itemize}
}


%\subsection{In preparation}
\cventry{}{In preparation and submitted}{}{}{}
{
  \begin{itemize}
  \item L. A. Souza, E. V. Chimanski, T. Frederico, B. V. Carlson, M. S. Hussein. Four-body eikonal approach to three-body halo nuclei scattering. (https://uk.arxiv.org/abs/1806.06278v1)
%  \item J. H. Alvarenga Nogueira, E. V. Chimanski. Two and three body problems for bound states with singular potential.
  \item E. V. Chimanski, B. V. Carlson, R. Capote, A J Koning. Extension to the Multi-Step Direct Model. 
  \item Manuel Schottdorf, Emanuel V. Chimanski and Ulf Dieckmann. Universality in evolution.
  \end{itemize}
}

%\cventry{}{Jornal Articles - Submitted}{}{}{}
%{
%	\begin{itemize}
%        \item Emanuel V. Chimanski, Erico L. Rempel, Roman Chertovskih. On-off intermittency and spatiotemporal chaos in three-dimensional Rayleigh-B{\'e}nard convection, \textit{Advances in Space Research}, 2015.
%        \end{itemize}
%}

%\cventry{}{Jornal Articles - Accepted for publication}{}{}{}
%{
%	\begin{itemize}
%        \item R. Chertovskih, E. V. Chimanski and E. L. Rempel. Route to hyperchaos in Rayleigh-B{\'e}nard convection, \textit{Europhysics Letters}, 2015 (http://arxiv.org/abs/1506.00693).
%        \end{itemize}
%}

%\subsection{Published}

%\subsection{Books and Chapters}
\cventry{}{Books and Chapters}{}{}{}
{
	\begin{itemize}
        \item Chimanski, E. V., Martins, C. G. L., Chertovskih, R., Rempel, E. L., Roberto, M., Caldas, I. L., Chian, A. C.-L. Intermittency and transport barriers in fluids and plasmas, In: From nonlinear dynamics to complex systems: A Mathematical modeling approach, Springer, Elbert E. N. Macau (Ed.), Springer. (https://doi.org/10.1007/978-3-319-78512-7\_5)
        \end{itemize}
}

%\cventry{}{Proceedings}{}{}{}
%{
%	\begin{itemize}
%        \item On-off intermittency and spatiotemporal chaos in three-dimensional Rayleigh-B{\'e}nard convection, \textit{Advances in Space Research}, 2015.
%        \item Route to hyperchaos in  Rayleigh-B{\'e}nard convection, \textit{ Europhysics Letters}, 2015 (http://arxiv.org/abs/1506.00693).
%        \end{itemize}
%}


\section{Others}
\cventry{}{Scientific Societies}{}{}{}
{
	\begin{itemize}
	\item Brazilian Society of Physics
	\item American Physical Society
        \end{itemize}
}
%\cventry{}{Awards}{}{}{}
%{
%	\begin{itemize}
%	\item Special mention, Committee of XIII International Workshop on Hadron Physics
%	\end{itemize}
%}

\cventry{}{Reviewer}{}{}{}
{
  \begin{itemize}
  \item Proceedings for the CNR$^{*}$18 published online and in print by Springer Nature.
  \item Communications in Nonlinear Science and Numerical Simulation.
   \item Brazilian Journal of Physics.
  \end{itemize}
}

\cventry{}{Invited Talks}{}{}{}
{
  \begin{itemize}
    \item Nuclear and Chemical Sciences Division (NACS), LLNL 2021.
  \item  Department of Physics of Fluminense Federal University Cariri – RJ/Brazil, 2020 
  \item Department of Physics of Federal University of Cariri – CE/Brazil, 2020  
  \item CEA, DAM, DIF, Bruyères-le-Châtel, France, June-2018
  \item Lawrence Livermore National Laboratory -- LLNL - Livermore/California -- US, September-2018.
  \item Department of Physics, Florida State University -- FSU - Tallahassee/Florida -- US , November-2018.
  \item Department of Physics and Astronomy Texas A\&M University - Commerce/Texas -- US, November-2018
  \end{itemize}
}


\cventry{}{Conferences, meetings and workshops. Talk $^{*}$ and poster $^{\dagger}$ contribution}{}{}{}
{
  \begin{itemize}
  \item Nuclear Data Week 2020 - CSEWG meeting. 2020.   
  \item Division of Nuclear Physics Meeting (DNP—APS) 2020.
    \begin{itemize}
      \item Improving Inelastic Scattering Descriptions: Reaction Theory for Deformed Targets with the QRPA $^{*}$.
      \end{itemize}
    \item Brazilian Meeting on Nuclear Physics 2020.
      \begin{itemize}
        \item Nucleon Induced Reactions Theory for Deformed Target Nuclei: Angular Momentum Restoration and the QRPA $^{*}$.
 \item Inclusive Emissions from 3He Breakup Reaction on Medium and Heavy Targets $^{\dagger}$.
        \end{itemize}
    \item  Far West Section Meeting (FWS – APS) 2020.
      \begin{itemize}
        \item  Combining State-of-Art Nuclear Structure Theory with Modern
Reaction Descriptions: Nucleon-Induced Reactions.
        \end{itemize}
    \item  2019 Fall Meeting of the APS Division of Nuclear Physic. October 14-17, 2019; Crystal City, Virginia.
    \begin{itemize}
      \item Improved Inelastic Scattering Descriptions for Nuclear Data Evaluations, Nuclear Structure and Reaction Studies $^{*}$.
      \end{itemize}
  \item 6th  International Workshop on Compound-Nuclear Reactions and Related Topics (CNR$^{*}$18), 2018.
     \begin{itemize}
      \item Multi-step direct reaction models including collectivity in nucleon induced reactions$^{*}$.
     \end{itemize}
  \item 15th International Conference on Nuclear Reaction Mechanisms, 2018.
     \begin{itemize}
      \item Statistical multi-step direct reaction models and the RPA$^{*}$.
     \end{itemize}
   \item XL Brazilian Meeting on Nuclear Physics, 2017.
     \begin{itemize}
      \item One- and two-step direct cross sections for nucleon-induced reactions$^{*}$.
      \item Reactions and structure of three-fragment weakly bound nuclei$^{\dagger}$.
     \end{itemize}
   \item Physics meeting, 2016.
     \begin{itemize}
     \item Quasi-Particle -- Quasi-Hole Nature of High Energy RPA Modes$^{\dagger}$.
     \end{itemize}
   \item 6th International Conference on Nonlinear Science and Complexity, 2016.
    \begin{itemize}
     \item Route to hyperchaos and Intermittency in Rayleigh-B{\'e}nard convection$^{*}$.
     \end{itemize}
   \item National Meeting of Statistical Physics, 2015.
     \begin{itemize}
     \item Leaking square quantum billiards$^{\dagger}$.
     \end{itemize}
   \item Tenth Latin American Conference on Space Geophysics, 2014.
     \begin{itemize}
     \item Route to hyperchaos in Rayleigh-B{\'e}nard convection$^{\dagger}$.
     \end{itemize}
   \item Brazilian National Meeting on Condensed Matter Physics, 2012.
     \begin{itemize}
     \item Influence of obtuse and acute angles in statistic of energy levels of quantum polygonal billiards$^{\dagger}$.
     \end{itemize}
   \item Physics meeting, 2011.
     \begin{itemize}
     \item Energy levels statistics in quantum obtuse triangular billiards$^{\dagger}$.
    \end{itemize}
  \end{itemize}
}
\end{document}
